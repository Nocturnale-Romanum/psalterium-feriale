% !TEX TS-program = lualatex
% !TEX encoding = UTF-8

\documentclass[psalterium-feriale.tex]{subfiles}

\ifcsname preamble@file\endcsname
  \setcounter{page}{\getpagerefnumber{M-pf10_temporale}}
\fi

\begin{document} %% THIS SUBFILE BEGINS ON AN ODD, RIGHT PAGE

\feast{PT}{Proprium de Tempore}{Proprium de Tempore}{Proprium de Tempore}{1}{}{}{}{}{}{}
\addcontentsline{toc}{chapter}{Proprium de Tempore}

\feast{A1F1}{Tempus Adventus}
	{Proprium de Tempore}{Tempus Adventus}{2}{}
	{}{}{}
	{}
	{}
\rubric{Dominicæ \Rnum{1} et \Rnum{2}}
\gscore{A1F1I}{I}{}{Regem venturum Dominum\idxnewline{}(in Dominicis)}
\rubric{In Feriis Hebdomadæ \Rnum{1} et \Rnum{2}}
\gscore{A1F2I}{I}{}{Regem venturum Dominum\idxnewline{}(in Feriis)}
\rubric{Dominicæ \Rnum{3} et \Rnum{4}}
\gscore{A3F1I}{I}{}{Prope est jam Dominus\idxnewline{}(in Dominicis)}
\pagebreak
\rubric{In Feriis Hebdomadæ \Rnum{3} et \Rnum{4}}
\gscore{A3F2I}{I}{}{Prope est jam Dominus\idxnewline{}(in Feriis)}
\gscore{A1H}{H}{}{Verbum supernum prodiens}

\feast{1224}{In Vigilia Nativitatis Domini}
	{Proprium de Tempore}{In Vigilia Nativitatis Domini}{3}{24 Decembris}
	{}{}{Jesu Christi, Domini nostri!Vigila Nativitatis}
	{}
	{}
\gscore{1224I}{I}{}{Hodie scietis}

\rubric{Si Vigilia venerit in Dominica, in \Rnum{3} Nocturno versus ut infra, alii de Dominica. Si venerit in Feria, versus ut infra.}

\versiculus{Hódie sciétis quia véniet Dóminus.}{Et mane vidébitis glóriam ejus.}

\feast{1225}{Tempus Nativitatis}
	{Proprium de Tempore}{Tempus Nativitatis}{2}{}
	{}{}{}
	{}
	{}
\rubric{In Officio de Tempore post Nativitatem, juxta rubricas Pii \textsc{pp} \Rnum{10}, omnia ut in Psalterio Festivo. Juxta rubricas Joannis \textsc{pp} \Rnum{23}, Antiphonæ, Psalmi et Versus ut in Psalterio Feriali, reliqua ut infra.}

\gscore{1225I}{I}{}{Christus natus est}
\gscore{1225H}{H}{}{Jesu redemptor omnium}

\feast{0102}{In Octava S. Stephani}
	{Proprium de Tempore}{Tempus Nativitatis}{3}{2 Januarii}
	{}{}{Stephani!Octava}
	{}
	{}
\gscore{1226I}{I}{}{Christum natum qui beatum}
\rubric{Hymnus \normaltext{Deus tuorum militum} pag.\ \pageref{UMEXHa}.}

\feast{0103}{In Octava S. Joannis Evangelistæ}
	{Proprium de Tempore}{Tempus Nativitatis}{3}{3 Januarii}
	{}{}{Stephani!Octava}
	{}
	{}
\rubric{Invitatorium \normaltext{Regem Apostolorum} pag.\ \pageref{APEXIb}.}\\
\rubric{Hymnus \normaltext{Æterna Christi munera} pag.\ \pageref{APEXH}.}

\feast{0104}{In Octava SS. Innocentium}
	{Proprium de Tempore}{Tempus Nativitatis}{3}{4 Januarii}
	{}{}{Innocentium!Octava}
	{}
	{}
\rubric{Invitatorium \normaltext{Regem Martyrum} pag.\ \pageref{UMEXIa}.}
\gscore{1228H}{H}{}{Audit tyrannus anxius}

\feast{0106}{Per Octavam Epiphaniæ}
	{Proprium de Tempore}{Post Epiphaniam}{3}{}
	{}{}{Innocentium!Octava}
	{}
	{}

\rubric{In Officio de Tempore per Octavam Epiphaniæ, juxta rubricas Pii \textsc{pp} \Rnum{10}, Antiphonæ et Psalmi ut in Festo. Juxta rubricas Joannis \textsc{pp} \Rnum{23}, Antiphonæ et Psalmi ut in Psalterio.}

\gscore{0106I}{I}{}{Christus apparuit nobis}
\gscore{0106H}{H}{}{Crudelis Herodes}

\feast{Q0F4}{Feria Quarta Cinerum}
	{Proprium de Tempore}{Feria Quarta Cinerum}{3}{}
	{}{}{Feria Quarta Cinerum}
	{}
	{}
\rubric{In hac et sequentibus Feriis omnia dicuntur ut in Psalterio per annum.}

\feast{Q1F1}{Tempus Quadragesimæ}
	{Proprium de Tempore}{Tempus Quadragesimæ}{2}{}
	{}{}{}
	{}
	{}
\thispagestyle{empty}
\gscore{Q1I}{I}{}{Non sit vobis vanum}
%\rubric{In feriis ponitur ad libitum tonus 7a.} %% probably not useful in a book that does not have the Venite tones
\gscore{Q1H}{H}{}{Ex more docti mystico}

\newpage

\feast{Q1F5}{Tempus Passionis}
	{Proprium de Tempore}{Tempus Passionis}{2}{}
	{}{}{}
	{}
	{}
\gscore{Q5I}{I}{}{Hodie si vocem}
\gscore{Q5H}{H}{}{Pange lingua... Proelium}

\feast{P0F1}{Tempus Paschale}
	{Proprium de Tempore}{Tempus Paschale}{2}{}
	{}{}{}
	{}
	{}
\gscore{P0I}{I}{}{Surrexit Dominus vere}
\gscore{P1H}{H}{}{Rex sempiterne Domine}

\feast{P5F5}{Per Octavam Ascensionis}
	{Proprium de Tempore}{Post Ascensionem}{2}{}
	{}{}{}
	{}
	{}
\rubric{Post Ascensionem, usque ad Vigiliam Pentecostes inclusive, juxta rubricas Pii \textsc{pp} \Rnum{10}, dicuntur in Officio de Tempore Antiphonæ, Psalmi et Versus Nocturnorum ut in Festo Ascensionis, et in Festis minoribus, Antiphonæ, Psalmi et Versus Nocturnorum ut in Psalterio Tempore Paschali. Juxta rubricas Joannis \textsc{pp} \Rnum{23}, dicuntur in Officio de Tempore et in Festis \Rnum{3} classis Antiphonæ et Psalmi ut in Psalterio, reliqua ut infra.}

\gscore{P5F5I}{I}{}{Alleluia Christum Dominum}
\gscore{P5F5H}{H}{}{Aeterne Rex altissime}

\rubric{Dominica, Feriæ II et V:}
\versiculus{Ascéndit Deus in jubilatióne, allelúia.}{Et Dóminus in voce tubæ, allelúia.}
\rubric{Feriæ II et VI:}
\versiculus{Ascéndens Christum in altum, allelúia.}{Captívam duxit captivitátem, allelúia.}
\rubric{Feria III et Sabbato:}
\versiculus{Ascéndo ad Patrem meum, et Patrem vestrum, allelúia.}{Deum meum, et Deum vestrum, alleluia.}


\feast{H1F1}{Per Octavam Sanctæ Trinitatis}
	{Proprium de Tempore}{Post Pentecosten}{3}{}
	{}{}{Trinitatis}
	{}
	{}
\rubric{Si de ea fit Officio cum Antiphonis et Psalmis Feriæ.}
\gscore{H1I}{I}{}{Deum verum unum}
\gscore{H1H}{H}{}{Summae Parens clementiae}

\newpage

\feast{H1F5}{Per Octavam Sanctissimi Corporis Christi}
	{Proprium de Tempore}{Post Pentecosten}{3}{}
	{}{}{Jesu Christi, Domini nostri!Octava Corporis}
	{}
	{}
\rubric{Si de ea fit Officio cum Antiphonis et Psalmis Feriæ.}
\gscore{H1F5I}{I}{}{Christum regem adoremus}
\gscore{H1F5H}{H}{}{Sacris solemniis}

\feast{H2F6}{Per Octavam Sacratissimi Cordis Jesu}
	{Proprium de Tempore}{Post Pentecosten}{3}{}
	{}{}{Jesu Christi, Domini nostri!Octava Cordis}
	{}
	{}
\rubric{Si de ea fit Officio cum Antiphonis et Psalmis Feriæ.}
\gscore{H2F6I}{I}{}{Cor Jesu amore}
\gscore{H2F6H}{H}{}{Auctor beate saeculi}

\end{document} 