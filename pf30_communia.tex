% !TEX TS-program = lualatex
% !TEX encoding = UTF-8

\documentclass[psalterium-feriale.tex]{subfiles}

\ifcsname preamble@file\endcsname
  \setcounter{page}{\getpagerefnumber{M-pf30_communia}}
\fi

\begin{document} % THIS SUBFILE BEGINS ON AN EVEN, LEFT PAGE

\feast{CO}{Commune Sanctorum}
	{Commune Sanctorum}{Commune Sanctorum}{1}{}{}{}{}{}{}
\addcontentsline{toc}{chapter}{Communia}

\feast{APEX}{Commune Apostolorum\\et Evangelistarum\\extra Tempus Paschale}
	{Commune Sanctorum}{Commune Apostolorum extra T. P.}{2}{}
	{}{}{}
	{}
	{}
\addcontentsline{toc}{section}{Commune Apostolorum}

\gscore{APEXIb}{I}{}{Regem Apostolorum\idxnewline(tonus festivus)}
\gscore{APEXH}{H}{}{Aeterna Christi munera\idxnewline{}(Apostolorum)}

\feast{APTP}{Commune Apostolorum\\et Evangelistarum\\Tempore Paschali}
	{Commune Sanctorum}{Commune Apostolorum et Evangelistarum T. P.}{2}{}
	{}{}{}
	{}
	{}
\gscore{APTPI}{I}{}{Regem Apostolorum\idxnewline(Tempore Paschali)}
\gscore{APTPH}{H}{}{Tristes erant Apostoli}
\rubric{Ab Ascensione ad Pentecosten, non dicitur \normaltext{Qui surrexísti a mórtuis} sed \normaltext{Qui scandis super sídera}.}

\feast{UMEX}{Commune Martyrum}
	{Commune Sanctorum}{Commune Martyrum}{2}{}{}{}{}{}{}
\addcontentsline{toc}{section}{Commune Martyrum}

\rubric{Tonus simplex:}
\gscore{UMEXIa}{I}{}{Regem Martyrum\idxnewline(tonus simplex)}
\rubric{Tonus festivus:}
\gscore{UMEXIb}{I}{}{Regem Martyrum\idxnewline(tonus festivus)}
\rubric{Tempore Paschali:}
\gscore{MRTPI}{I}{}{Exsultent in Domino}
\rubric{Pro uno, tonus prior:}
\gscore{UMEXHa}{H}{}{Deus tuorum militum\idxnewline(tonus prior)}
\rubric{Pro uno, tonus alter:}
\gscore{UMEXHb}{H}{}{Deus tuorum militum\idxnewline(tonus alter)}
\rubric{Pro plurimis:}
\gscore{PMEXH}{H}{}{Aeterna Christi munera\idxnewline{}(Martyrum)}
\rubric{Sic semper terminandus est Hymnus prædictus.}

\feast{CONP}{Commune Confessoris}
	{Commune Sanctorum}{Commune Confessoris}{2}{}{}{}{}{}{}
\thispagestyle{empty}
\addcontentsline{toc}{section}{Commune Confessoris}

\rubric{Tonus simplex:}
\gscore{COPOIa}{I}{}{Regem Confessorum\idxnewline(tonus simplex)}
\rubric{Tonus festivus:}
\gscore{COPOIb}{I}{}{Regem Confessorum\idxnewline(tonus festivus)}
\rubric{Tempore Paschali:}
\gscore{COPOId}{I}{}{Regem Confessorum\idxnewline(Tempore Paschali)}

\rubric{Feria \Rnum{2}:}
\gscore{COPOF2H}{H}{}{Iste Confessor (F2)}
\rubric{Feria \Rnum{3}:}
\gscore{COPOF3H}{H}{}{Iste Confessor (F3)}
\rubric{Feria \Rnum{4}:}
\gscore{COPOF4H}{H}{}{Iste Confessor (F4)}
\rubric{Feria \Rnum{5}:}
\gscore{COPOF5H}{H}{}{Iste Confessor (F5)}
\rubric{Feria \Rnum{6}:}
\gscore{COPOF6H}{H}{}{Iste Confessor (F6)}
\rubric{Sabbato:}
\gscore{COPOF7H}{H}{}{Iste Confessor (Sabb)}

\feast{MU}{Commune Virginum aut non Virginum}
	{Commune Sanctorum}{Commune Virginum aut non Virginum}{2}{}{}{}{}{}{}
\addcontentsline{toc}{section}{Commune Virginum aut non Virginum}
\thispagestyle{empty}

\rubric{Pro Virgine, extra Tempus Paschale, tonus simplex:}
\gscore{MUVXIa}{I}{}{Regem Virginum\idxnewline(tonus simplex)}
\rubric{Pro Virgine, extra Tempus Paschale, tonus festivus:}
\gscore{MUVXIb}{I}{}{Regem Virginum\idxnewline(tonus festivus)}
\rubric{Pro Virgine, Tempore Paschali:}
\gscore{MUVXId}{I}{}{Regem Virginum\idxnewline(Tempore Paschali)}
\rubric{Pro una non Virgine:}
\gscore{MUNXIa}{I}{}{Laudemus\idxnewline(unius non Virginis)}
\rubric{Pro plurimis non Virginis:}
\gscore{MUNXIb}{I}{}{Laudemus (plurimarum\idxnewline{}non Virginum)}
\rubric{Pro Virgine non Martyre, strophæ 1, 4 et 5 tantum dicuntur.\\Pro non Virgine, strophæ 4 et 5 tantum dicuntur.}
\gscore{MUVMHa}{H}{}{Virginis proles\idxnewline{}(tonus prior)}
\rubric{Alter tonus ad libitum:}
\gscore{MUVMHb}{H}{}{Virginis proles\idxnewline{}(tonus alter)}

%%%% UNCOMMENT THIS IF WE CONSIDER THAT DAYS WITHIN OCTAVES CAN BE SIMPLEX

% \feast{CDED}{Commune Dedicationis Ecclesiæ}
	% {Commune Sanctorum}{Commune Dedicationis Ecclesiæ}{2}{}{}{}{}{}{}
% \addcontentsline{toc}{section}{Commune Dedicationis}

% \gscore{CDEDI}{I}{}{Domum Dei decet}
% \gscore{CDEDH}{H}{}{Coelestis urbs Jerusalem}


\feast{CBMV}{Commune Beatæ Mariæ Virginis}
	{Commune Sanctorum}{Commune Beatæ Mariæ Virginis}{2}{}{}{}{}{}{}
\addcontentsline{toc}{section}{Commune Beatæ Mariæ Virginis}

\gscore{CBMVI}{I}{}{Sancta Maria}
\rubric{Hymnus \normaltext{Quem terra pontus æthera}, ut in Officio de S. Maria in Sabbato, infra.}

\feast{CSMS}{De Sancta Maria in Sabbato}
	{Commune Sanctorum}{De Sancta Maria in Sabbato}{2}{}
	{}{}{Sabbato Sanctae Mariae}
	{}{}
\addcontentsline{toc}{section}{De Sancta Maria in Sabbato}
\thispagestyle{empty}

\gscore{CSMSI}{I}{}{Ave Maria}
\gscore{CBMVH}{H}{}{Quem terra pontus aethera}

\end{document}