% !TEX TS-program = lualatex
% !TEX encoding = UTF-8

\documentclass[psalterium-feriale.tex]{subfiles}

\ifcsname preamble@file\endcsname
  \setcounter{page}{\getpagerefnumber{M-pf01_prolegomena}}
\fi

\begin{document}

\begin{titlepage}
\begin{center}
\null\vspace{5mm}
{\Large\sc{}Nocturnale Romanum}

\vspace{5mm}

{\large\sc{}Tomus IIs}

\vspace{3.5cm}

{\Huge{}PSALTERIUM}

\vspace{1cm}

{\Large\sc{}pro nocturnis horis\\in dominicis, feriis et festis minoribus}

\vspace{5mm}

{\large\sc{}secundum ordinem Divini Officii\\a Pio pp X restituti}

\vspace{5mm}

{\large\sc{}cum Antiphonis, Invitatoriis et Hymnis antiquis\\in cantu gregoriano}


\vfill

MMXXV

\end{center}
\end{titlepage}

\feast{OR}{Proœmium}
	{Proœmium}{Proœmium}{2}{}{}{}{}{}{}
\thispagestyle{empty}
\addcontentsline{toc}{section}{Proœmium}

\begin{paracol}[1]*{2}

\begin{english}
	
\intermediatetitle{Unfinished business}

{\setlength{\parindent}{5mm}\small

In 1912, Pope Saint Pius X promulgated the \emph{editio typica} of the \emph{Antiphonale Romanum} for the day hours of Divine Office, 
four years after having published the \emph{Graduale Romanum} for the Mass. 

This should have been the second of three steps towards the
restoration of Gregorian Chant in the Roman liturgy. Unfortunately, it would be the last. 
Work on a \emph{Liber Nocturnalis} had begun at Solesmes in the 1880s, and the Vatican commissioned the monks to work towards its publication. 
In 1901, they had been forced out of their abbey into what would become Quarr Abbey on the Isle of Wight in southern England.
It was in these dire conditions that Dom Mocquereau and his team worked until 1922.

Working on the responsories entailed collecting photographies of relevant manuscripts, transcribing them into comparative tables, and 
establishing critical versions of the melodies. But most of the antiphons of the weekly psalter of the Roman Office had no precedent
in the manuscript tradition, as they had been newly created by Saint Pius X's 1911 reform of the Breviary. 

The monks composed the music 
for these new antiphons, based on existing ones, adhering to them more or less closely depending on the adaptations needed to fit the text. 
In early 1916, they sent their work to the \emph{Typis Polyglottis Vaticanis}, and after a few iterations, the weekly Matins psalter was deemed fit for publication.

Using the same process, the Proper of Seasons for Advent and Christmas was finished in 1920, and Holy Week in 1922.

Around 1922, for a reason that the Solesmes archives have yet to reveal, Vatican funding was interrupted and the whole project was shelved.
Only the Office for Holy Week, including four days of the weekly psalter and the responsories for Palm Sunday and Tenebræ, was published, in 1922.

A proof copy of the weekly psalter, sent by the Vatican printer to Dom Mocquereau in June 1916 for 
final proofreading, was rediscovered in December 2023 in the Solesmes archives, by Dom Jacques-Marie Guilmard.

The antiphons present in this book are taken from this proof copy, for those antiphons that do not exist in the manuscript tradition. 
Those that do exist in the manuscript tradition, as well as the invitatories, have received newly revised, critically informed melodies.
Hymns are taken from the 2019 \emph{Liber Hymnarius} with their text antecedent to Urban VIII's reform of the hymns, according to indults 
given to this effect by Saint Pius X.

This book is has now received thorough proofreading and, while it is certainly not perfect, nor approved by any authority of the Church, the editors team feels that it is good enough for stable use. 
You can still provide feedback at\\{\footnotesize\url{https://github.com/Nocturnale-Romanum/nocturnale-romanum/issues}}

}

\end{english}

\switchcolumn

\intermediatetitle{Un goût d'inachevé}

{\setlength{\parindent}{5mm}\small

En 1912, le pape saint Pie X promulguait l'\emph{editio typica} de l'\emph{Antiphonale Romanum} pour les heures diurnes de l'Office divin, 
quatre ans après avoir publié le \emph{Graduale Romanum} pour le chant de la Messe. 

Cela aurait dû être la deuxième de trois étapes vers 
la restauration du chant grégorien dans la liturgie romaine: malheureusement, ce serait la dernière.
Le travail sur un \emph{Liber Nocturnalis} avait commencé à Solesmes dans les années 1880, et le Vatican subventionna cette activité
en vue de sa publication. En 1901, les moines, chassés de leur abbaye, avaient trouvé refuge dans ce qui deviendrait l'abbaye de Quarr 
sur l'île de Wight, au sud de l'Angleterre. C'est dans ces conditions précaires que dom Mocquereau et son équipe travaillèrent jusqu'en 1922.

Travailler sur les répons consistait à collecter des photographies de manuscrits, les transcrire dans des tableaux comparatifs pour enfin
établir une version critique de leur mélodie. Mais la plupart des antiennes du psautier hebdomadaire de l'Office romain étaient absentes
des manuscrits, car elles avaient été créées lors de la réforme du Bréviaire par saint Pie X en 1911. 

Les moines composèrent donc la 
mélodie de ces nouvelles antiennes, en se basant sur des antiennes du répertoire, de manière plus ou moins fidèle en fonction des adaptations 
nécessaires pour adhérer au nouveau texte. Début 1916, ils envoyèrent leur travail à la \emph{Typis Polyglottis Vaticanis}, et après
quelques allers-retours, le travail fut terminé.

De la même manière, le Propre du Temps pour l'Avent et Noël fut achevé en 1920, et la Semaine Sainte en 1922.

Vers 1922, pour une raison que les archives de Solesmes n'ont pas encore révélé, le financement romain fut interrompu, et l'ensemble du projet 
fut stoppé. Seul l'office de la Semaine Sainte, dont quatre jours du psautier hebdomadaire, et les répons pour les Rameaux et les Ténèbres, fut 
publié, en 1922.

Les épreuves du psautier hebdomadaire, envoyées par les éditions vaticanes à dom Mocquereau en juin 1916 pour relecture finale, furent
redécouvertes en décembre 2023 dans les archives de Solesmes par dom Jacques-Marie Guilmard. 

Les antiennes présentes dans ce livre sont tirées de ces épreuves, pour les antiennes qui n'existent pas dans la tradition manuscrite.
Celles qui y figurent, ainsi que les invitatoires, ont reçu des mélodies nouvellement révisées de manière critique.
Les hymnes ont été pris au \emph{Liber Hymnarius} de 2019, avec leur texte antérieur à la réforme des hymnes d'Urbain VIII, selon les 
indults accordés à cet effet par saint Pie X. 

Ce livre a bénéficié d'une relecture approfondie, et, bien qu'il ne soit ni parfait, ni approuvé pour l'usage liturgique par l'autorité de l'Église, ses éditeurs estiment qu'il est assez mûr pour un usage pérenne.
Vous pouvez toujours signaler les corrections nécessaires sur\\{\footnotesize\url{https://github.com/Nocturnale-Romanum/nocturnale-romanum/issues}}


}

\switchcolumn*

\begin{english}

\intermediatetitle{On the rhythm of Gregorian Chant\\as considered in this edition}

{\setlength{\parindent}{5mm}\small

This psalter is intended for all those who wish to sing the night hour of Divine Office, 
according to one of its Roman preconciliar variants, whatever their approach to rhythm.

To this end, the editors wish to briefly expose the principles that have informed 
the correspondence between neumatic notation and horizontal \emph{episemata} in this book.

In Gregorian Chant, notes have a base length, 
henceforth the \emph{syllabic value}. 
It is the value of a syllable sung on a single note, when cantillation has the text sung on one note for each syllable.

When the neume for a syllable sung on a single note receives an \emph{episema}, or a letter T (\emph{tenete}), 
this note takes on the \emph{long value}, that is longer than the syllabic value. 
In this case, the square note receives an episema as well in this edition.

Notes within a melisma (that is, a syllable sung on more than one note) receive by default the \emph{short value}, 
a value that is shorter than the syllabic value. 
Frequently, their neumes receive an \emph{episema}, or an angled shape distinct from the usual one, or a neumatic break where the complex neume disintegrates into several simpler shapes, 
in which case the notes receive the syllabic value.
In this case, the square notes receive an episema as well, except notes before a \emph{quilisma}: 
it is common knowledge that those are to be somewhat lengthened even if they are not marked with an \emph{episema}.

{\gresetnabc{1}{visible}
\gresetclef{invisible}
\gresetinitiallines{0}
\begin{center}
\begin{minipage}[c]{0.7\textwidth}
\gregorioscore{\subfix{nocturnale-romanum/gabc/rhythmica_en}}
\end{minipage}
\end{center}
}

~

The proportion between the \emph{short value} and the \emph{syllabic value}, 
and between the \emph{syllabic value} and the \emph{long value}, should be consistent, but is at the cantor's discretion. 

It should also take into account the nature of the note: a \emph{stropha} marked with an \emph{episema} receives a 
syllabic value somewhat shorter than that of a \emph{virga} also marked with an \emph{episema}, 
the \emph{stropha} itself being slightly shorter than the \emph{virga}.

Finally, the end of a musical or textual phrase naturally brings about a certain lengthening of the notes all while the sound diminishes. 
This is customarily indicated by the \emph{punctum mora} or \emph{mora} dot.

}

\end{english}

\switchcolumn

\intermediatetitle{Le rythme du chant grégorien\\tel qu'il a servi à préparer cette édition}

{\setlength{\parindent}{5mm}\small

Ce psautier s'adresse à toutes les personnes qui désirent chanter l'heure nocturne de l'Office divin 
selon l'une de ses variantes romaines pré-conciliaires --- quelle que soit leur approche du rythme.

Pour cela, les éditeurs souhaitent exposer brièvement les principes qui, dans cet ouvrage, 
ont informé la correspondance entre la notation neumatique antique et l'usage désormais bien connu des épisèmes horizontaux.   

Dans le chant grégorien, les notes ont une longueur de base, qu’on peut appeler \emph{valeur syllabique}. 
Celle-ci est la valeur d’une syllabe chantée sur une seule note --- lorsque le texte de la mélodie est cantillé, une note sur chaque syllabe. 

Quand le signe neumatique associé à une syllabe chantée sur une seule note, a reçu dans les manuscrits un épisème, ou la lettre T (\emph{tenete}), 
cette note a une valeur longue, plus longue que la valeur syllabique. 
Dans notre publication, un épisème horizontal est ajouté à la note carrée.

Au sein d’un mélisme, c’est-à-dire au sein d’un enchaînement de plusieurs notes sur la même syllabe, 
les notes ont une valeur plus courte que la valeur syllabique. 
Souvent, les signes neumatiques qui les transcrivent, ont reçu dans les manuscrits un épisème, 
ou bien leur forme a été modifiée par rapport à l’usage habituel, ou encore on constate une «coupure» neumatique; 
dans ces cas, les notes concernées ont une valeur syllabique, 
et notre publication ajoute des épisèmes horizontaux aux notes carrées concernées, 
sauf pour les notes qui précèdent un quilisma, dont on sait qu’elles sont toujours légèrement allongées.

{\gresetnabc{1}{visible}
\gresetclef{invisible}
\gresetinitiallines{0}
\begin{center}
\begin{minipage}[c]{0.8\textwidth}
\gregorioscore{\subfix{nocturnale-romanum/gabc/rhythmica_fr}}
\end{minipage}
\end{center}
}

~

Les rapports entre la \emph{valeur courte} et la \emph{valeur syllabique}, 
et entre la \emph{valeur syllabique} et la \emph{valeur longue}, doivent être cohérents entre eux, mais sont à la main du chantre en fonction de l'acoustique du lieu.

La nature des notes devrait être aussi prise en compte: 
une \emph{stropha} épisémée prend une valeur syllabique plus légère que celle d'une \emph{virga} épisémée, 
la \emph{stropha} étant par nature plus légère que la \emph{virga}.

Enfin, le chant s'élargit naturellement, tout en diminuant de volume, à la fin d'une phrase textuelle ou musicale. C'est ce qui est indiqué, selon la coutume, par le point \emph{mora}.

}

\switchcolumn*
\begin{english}

\intermediatetitle{On critical restitution}

{\setlength{\parindent}{5mm}\small

This book is decidedly a practical edition. It attempts to be critically informed in the following ways.

Antiphons and invitatories are annotated with sangallian neumes, either from the Hartker antiphonary, or adapted from existing pieces to fit 
the newly composed ones. In this case, they are given between square brackets.

In some rare cases, when the manuscripts agree on a text that is not markedly different from that found in the Roman Breviary, this text was used instead of that found in the Breviary, as was done for the day hours in the 1912 Roman Antiphonary.

The assignment of particular Hymn tones to particular days 
(like the six tones of \emph{Iste Confessor} given for minor feasts throughout the six days of the week), 
the assignment of particular Invitatory tones to particular categories of saints 
(like the solemn tone of \emph{Regem Confessorum} for male Doctors of the Church), and the assignment of particular \emph{Venite} tones to invitatories, 
are mere proposals and purely indicative.

}

\end{english}

\switchcolumn

\intermediatetitle{Restitution et démarche critique}

{\setlength{\parindent}{5mm}\small

Cet ouvrage est fondamentalement une édition pratique. Il adopte une démarche critique de deux manières:

Les antiennes et invitatoires sont ornés de neumes sangalliens, soit issus de l'antiphonaire de Hartker, soit adaptés à partir de formules
existantes pour correspondre à la mélodie des nouvelles pièces. Dans ce cas, ils sont donnés entre crochets.

Dans quelques cas, quand les manuscrits donnent tous un texte qui n'est que légèrement différent du texte du bréviaire romain, ce texte a été conservé au lieu de celui du bréviaire, comme ce fut le cas pour les heures diurnes dans l'antiphonaire romain de 1912.

L'attribution de tons spécifiques des hymnes et des invitatoires à certains jours (par exemple les différents tons du \emph{Iste Confessor} en fonction du jour de la semaine, ou le ton solennel du \emph{Regem Confessorum} aux confesseurs docteurs de l'Église, 
ou encore l'attribution de tons du \emph{Venite} à chaque invitatoire), est une simple proposition,
purement indicative.

}

\switchcolumn*

\vfill
\sep
\vspace{8\baselineskip}
\pagebreak

\switchcolumn

\vfill
\sep
\vspace{8\baselineskip}
\pagebreak

\end{paracol}

\null\vfill

\intermediatetitle{Tabella neumatum}

{\gresetnabc{1}{visible}
\gresetclef{invisible}
\gresetinitiallines{0}
\gregorioscore{\subfix{nocturnale-romanum/gabc/neumata}}
}



\end{document}