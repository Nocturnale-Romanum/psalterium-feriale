% !TEX TS-program = lualatex
% !TEX encoding = UTF-8

\documentclass[psalterium-feriale.tex]{subfiles}

\ifcsname preamble@file\endcsname
  \setcounter{page}{\getpagerefnumber{M-pf40_sanctorale}}
\fi

\begin{document} %% THIS SUBFILE BEGINS ON A RIGHT, ODD PAGE

\feast{PS}{Proprium Sanctorum}
	{Proprium Sanctorum}{Proprium Sanctorum}{1}{}{}{}{}{}{}
\addcontentsline{toc}{chapter}{Proprium Sanctorum}

\rubric{In omnibus Festis novem Lectionum Domini, beatæ Mariæ Virginis, Angelorum, sancti Joannis Baptistæ, sancti Joseph, Apostolorum, Evangelistarum, necnon in omnibus Duplicibus \Rnum{1} et \Rnum{2} classis, Invitatorium, Antiphonæ, Psalmi et Versus Nocturnorum ut in Psalterio Festivo.}

\rubric{In reliquis Festis, Invitatorium et Hymnus ut in Proprio vel Communi, Antiphonæ, Psalmi et Versus Nocturnorum ut in Psalterio Feriali.}

\feast{0104}{In Octava S. Stephani}
	{Proprium Sanctorum}{Festa Januarii}{2}{2 Januarii}
	{}{}{Stephani!Octava}
	{}
	{}
\gscore{1226I}{I}{}{Christum natum qui beatum}

\feast{0104}{In Octava SS. Innocentium}
	{Proprium Sanctorum}{Festa Januarii}{2}{4 Januarii}
	{}{}{Innocentium!Octava}
	{}
	{}
\gscore{1228H}{H}{}{Audit tyrannus anxius}

\feast{0130}{S. Martinæ Virginis et Martyris}
	{Proprium Sanctorum}{Festa Januarii}{2}{30 Januarii}
	{}{}{Martinæ}
	{}
	{}
\gscore{0130H}{H}{}{Martinae celebri}

\feast{0212}{Ss. Septem Fundatorum\\Ordinis Servorum B. M. V. Confessorum}
	{Proprium Sanctorum}{Festa Februarii}{2}{12 Februarii}
	{}{}{Septem Fundatorum Ordinis Servorum B. M. V.}
	{}
	{}
\gscore{0212H}{H}{}{Bella dum late furerent}

\feast{0413}{S. Hermenegildi Martyris}
	{Proprium Sanctorum}{Festa Aprilis}{2}{13 Aprilis}
	{}{}{Hermenegildi}
	{}
	{}
\rubric{Si sequens Hymnus in primis Vesperis dictus erit, Hymnus \normaltext{Nullis te genitor} ut in Antiphonali.}
\gscore{0413H}{H}{}{Regali solio fortis Iberiae}

\feast{0518}{S. Venantii Martyris}
	{Proprium Sanctorum}{Festa Maii}{2}{18 Maii}
	{}{}{Venantii}
	{}
	{}
\gscore{0518H}{H}{}{Athleta Christi nobilis}

\feast{0619}{S. Julianæ de Falconeriis Virginis}
	{Proprium Sanctorum}{Festa Junii}{2}{19 Junii}
	{}{}{Julianæ de Falconeriis}
	{}
	{}
\gscore{0619H}{H}{}{Caelestis Agni nuptias}

\newpage
\feast{0707}{Ss. Cyrilli et Methodii Pont. et Conf.}
	{Proprium Sanctorum}{Festa Julii}{2}{7 Julii}
	{}{}{Cyrilli et Methodii}
	{}
	{}
\gscore{0707H}{H}{}{Sedibus caeli nitidis}

\feast{0708}{S. Elisabeth Lusitaniæ Reginæ Viduæ}
	{Proprium Sanctorum}{Festa Julii}{2}{8 Julii}
	{}{}{Elisabeth Lusitaniæ}
	{}
	{}
\gscore{0708I}{I}{}{Laudemus... Elisabeth}
\gscore{0708H}{H}{}{Domare cordis impetus}

\newpage
\feast{0917}{In Impressione Ss. Stigmatum\\S. Francisci Confessoris}
	{Proprium Sanctorum}{Festa Septembris}{2}{17 Septembris}
	{}{}{Francisci!Stigmata}
	{}
	{}
\gscore{0917H}{H}{}{Iste Confessor... vulnera}

\feast{1015}{S. Teresiæ Virginis}
	{Proprium Sanctorum}{Festa Octobris}{2}{15 Octobris}
	{}{}{Teresiæ}
	{Commune aut Virginum aut non Virginum, pag.\ \pageref{M-MU}, præter Hymnus.}
	{}
\gscore{1015H}{H}{}{Regis superni nuntia}

\feast{1020}{S. Joannis Cantii Confessoris}
	{Proprium Sanctorum}{Festa Octobris}{2}{20 Octobris}
	{}{}{Joannis Cantii}
	{}
	{}
\gscore{1020H}{H}{}{Corpus domas jejuniis}

\end{document}
